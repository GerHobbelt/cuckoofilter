\documentclass[11pt,letterpaper]{article}
\usepackage[margin=0.75in, bottom=1.25in]{geometry}
%\linespread{1.5}
\usepackage[T1]{fontenc}
\usepackage[bitstream-charter]{mathdesign}
%\usepackage{amsfonts}
%\usepackage{amssymb}
\usepackage{cite}
\emergencystretch 1em

\begin{document}
\title{Tail Filters}
\author{Jim Apple}
\maketitle

\section{Introduction}

We will discuss {\em approximate membership query} (or {\em ``AMQ''}) data structures.
An instance of an AMQ is an over-approximation of a set $S$, and is equipped with a {\bf lookup} operation that returns {\bf true} for any key in $S$ and {\bf false} with high probability for any key not in $S$.

\section{Notation}

If not otherwise specified, $U$ is a finite universe of keys, $F$ is an AMQ, $x \in U$ is a key, $s$ is a bitstring, $m$, $n$, and $i$ are non-negative integers, and $0 < \epsilon < 1$.
The size of a set $Y$ is denoted ``$|Y|$''; the length of a bitstring $s$ is also denoted $|s|$.
The binary logarithm $\log_2$ is denoted ``$\lg$''.
$\lfloor r \rfloor$ is the largest integer less than or equal to $r$.
%$\widehat{n}$ is $\{0, 1, \dots , 2^n - 1\}$, which is isomorphic with the set of bitstrings of length $n$.

Any function $f: P \to Q$ can be extended to a function from subsets of $P$ to subsets of $Q$ by $f(P) \triangleq \{ q \in Q: \exists p \in P, f(p) = q\}$.

$[n]$ denotes $\{0, 1, \dots n - 1 \}$.
Note that $[2^n]$ is trivially isomorphic to bitstrings of length $n$.
In an abuse of notation, when $n$ is implicit, values in $[2^n]$ and their equivalent bitstrings will be interchangable.
Similarly, ``$[2^n]$'' will sometimes denote the set of bitstrings of length $n$, rather than $\{0, 1, \dots, 2^n - 1\}$.

%% ; we'll call this isomorphism $v$.
%% The set of bitstrings of length $n$ can thus be denoted ``$v^{-1}[2^n]$''.
%% In an abuse of notation, $v^{-1}[2^n]$ will sometimes be denoted simply ``$[2^n]$''.

%% There also exists an injection, $w$, from the set of bitstrings $\cup_{i \in [n]} [2^i]$ to $[2^n]$ by $w(m \in [2^i]) = v(2^i + v^{-1}(m))$.
%% This representation is not surjective due to a single value of $[2^n]$ that has no pre-image: the bitstring consisting of all zeros

The set of bitstrings of length less than $n$,  $\bigcup_{i < n} [2^i]$, is isomorphic to $[2^n]$ via $(m \in [2^i]) \mapsto 2^i + m$.
This representation is not surjective due to a single value of $[2^n]$ that has no pre-image: $0$.
This value can be repurposed so that any value in $[2^n]$ can be thought of either as being absent or a bitstring of length less than $n$.

Each AMQ $F$ represents a subset of $U$.
The subset of $U$ represented by a structure $F$ is denoted $[F]$.
The size of the subset relative to the whole universe, $|[F]|/|U|$, is called the {\em false positive probability}, or {\em ``fpp''}.

The AMQ's discussed here support at least the following three operations:

\begin{description}
\item[initialize$(m, \epsilon)$] creates a structure.
  $m$ is the initial pre-allocated space of the data structure, such that $m$ calls to {\bf insert} can be made before more space needs to be allocated;
  $\epsilon$ is the initial fpp maximum, such that $m$ calls to {\bf insert} can be made before the fpp exceeds $\epsilon$.
\item[insert$(x)$] adds the key $x$ to $F$ such that $x \in [F]$.
  It might increase the fpp or the space usage of $F$, or both.
\item[lookup$(x)$] returns {\bf true} if $x \in [F]$.
  In the reverse direction, $[F]$ can be defined as $\{x \in U : {\bf insert}(x) = {\bf true}\}$.
  Note that if $x \in [F]$, it need not have been {\bf insert}ed, since $[F]$ is an over-approximation of the set of insert keys.
\end{description}

Using low-overhead dictionaries like the one described below, any subset of $U$ of size $n$ can be stored exactly (i.e. with fpp $n/U$) in $O(n \lg (|U|/n))$.
As such, we'll assume all calls to {\bf initialize} have $\epsilon > 1/|U|$.

Each key in the universe $U$ is associated with a bitstring of length $\lceil 2 \lg |U| \rceil$.
%; this is commonly done through hashing.
 % $\lg |U| - \lg \epsilon + c_1$, for some constant $c_1$ to be specified later.
For the remainder of the paper, we will assume that the keys inserted have random values.
This is frequently done by considering operations on the set $h(U)$, where $h$ is a suitable hash function, rather than $U$.
For the rest of this paper, the universe $F$ is over will actually be $h(U)$, though we will refer to it as $U$, as we have no more need for the original ``$U$''.

The main challenge of this work is to keep all of the following low:

\begin{enumerate}
\item The fpp of $F$
\item The number of bits needed to store $F$
\item The time complexities of {\bf insert} and {\bf lookup}
\end{enumerate}

\section{Practical quotienting dictionaries with continual minimal overhead}

%% [Combination of cuckoo filters, the A-B design of shingle filters or the journal version of Morton filters, and DySECT~\cite{maier2019dynamic}]

In this section, we'll show a practical data structure that can store a multi-map from bitstrings to bitstrings.
The basic structure follows that of dynamic space efficient cuckoo tables~\cite{maier2019dynamic}, which shows the space can be as low as 3\% more than the minimum while {\bf lookup} inspects no more than three cache lines.
In their experiments, {\bf lookup}s take 100ns on average and {\bf insert}s take 1500ns, even when the filter is 98\% full.

The central idea of~\cite{maier2019dynamic} is to store not one cuckoo hash table of a certain size, but several sub-tables of varying size.
The alternate slots of a key may be in distinct sub-tables.
When the map is near capacity, one of the sub-tables is doubled in capacity.
Since this doesn't double the space usage of whole table, the space usage increases by much less than the 100\% increase when doing traditional hash table size doubling.
The {\bf insert} operation will often succeed only by finding a path of keys to kick that ends in a sub-table recently doubled in capacity.
%Increasing the number of sub-tables decreases the maximum number of empty slots needed in order to accomodate a series of {\bf insert}s calls
%; \cite{maier2019dynamic} uses 256 tables, buckets of size 8, and 3 possible buckets per key.


Let $s \geq 0$ be an integral number of shingle bits, following~\cite{lehman20093}; each slot wil store $s$ bits to indicate its offset, simulating buckets of size $2^s$.
%Let $d > 1$ be an integral number of directories.
%Let $w > 0$ be an integral number of ``directory bits''; there are $2^w$ sub-tables in each directory.
Let $\alpha > 1$ be the maximum number of slots, as a factor of $n$, the number of key-value pairs.
A low $\alpha$ is desirable, but this is a derived, rather than a specified constant, and is limited by the constraints of the cuckoo hypergraph.

We store $d$ directories, each containing $2^w$ pointers to sub-tables.
Each sub-table has a capacity that is a power of two, and there are at most two different sub-table capacities present in the directory at one time.
If there are two different capacities, they differ by a factor of two.

Let $k$ be the bitlength of the keys and $v$ the bitlength of the values.
Let $2^l$ be the smaller sub-table capacity.
Then the capacity of the whole multi-map is between $2^{l+w}d$ (when all of the tables have the same capacity) and $2^l + 2^{l+1}(d 2^w - 1)$ (when all other tables have larger capacity).
Since the capacity is also between $n$ and $\alpha n$, we have $\lg (n/d) - 1 \leq l + w \leq \lg(n/d) + \lg \alpha$.

$k$ must be at least $\lg(N/d) + \lg \alpha$, where $N$ is the maximum number of elements that will be inserted into the map.
In a sub-table with $2^l$ slots, the slots have bitlength $s + v + k - w - l$.
This is at least $s + v$, wince $k \geq \lg(|U|/d) + \lg \alpha \geq l + w$.

%Since the the number of slots is between $n$ and $\alpha n$, since

We map from $n \geq d 2^{s+w}$ random keys that are $k$ bits in length to $v$-bit values (excluding the value $0$).
The value $0$ is used to indicate an empty slot.
%Note that $\lg n - (2w + 1) \lg d \leq l \leq \lg \alpha + \lg n - (2w - 2)\lg d$, since there are at least $n$ slots and at most $\alpha n$ slots.
%Each sub-table is either ``small'' or ``large''.
%Small sub-tables have $2^{\lceil \lg (\alpha n / d) \rceil - w - 1}$ slots, while large sub-tables have twice that.
%There must be at least one ``small'' sub-table.

%In the smaller subtables, the slots store values of length $3 + k + v - \lceil \lg (1.32 n) \rceil$, while in the larger tables they store values with one less bit.
%The total number of slots is kept between $n$ and $\alpha n$.
%This is always possible since increasing a table from large to small adds $2^{\lceil \lg (1.32 n) \rceil - 7} < 2^{\lg (1.32 n) - 6} = 1.32n/6 = 0.22n$

Let $(x,y)$ be a key-value pair.
We will state under what conditions $(x,y)$ is said to be a member of the set.
%The following definitions will help to describe the conditions under which $(x,y)$ is said to be present in the dictionary:
Let $x_0$ be the $w$ least significant bits of $x$, and $x_{1+}$ be $\lfloor x / 2^w \rfloor$.
In the first directory, let $l$ be the integer such that sub-table number $x_0$ has $2^l$ slots.
Let $x_2$ be $\lfloor x_{1+} / 2^l \rfloor$ and let $x_1$ be $x_{1+} - x_2 2^l$.
$x_1$ has bitlength $k - w - l$.

Let $z_0 = x_1$ and $y_0 = y 2^{k - l - w} + x_2$.

The bitlength of $y_0$ is $v + k - w - l$.

%% For each case, we'll construct a set of slot-value pairs.
%% The set will have size $O(2^s)$.
%% $(x,y)$ is a member of the multi-map when there is a slot-value pair $(p,q)$ such that slot $p$ contains value $q$.

%% Since $k \geq w + l - O(1)$, $x_1$
%% Since $x$ has length $k$, we know that $\lg x_2 < k - l - w$.
%% Let $y_0$ be $y 2^{k - l - w} + x_2$.

Let $x^i = (x + h_i(y)) \bmod 2^k$ for $0 < i < w$, and derive $x^i_0$, $l_i$, $z_i$, and $y_i$ from $x^i$.
%Follow the above pattern to derive $z^i$, $l^i$, $x^i_1$, $x^i_2$, and $y^i_0$.

The pair $(x,y)$ is said to be present in the dictionary if there exists $0 < i < d$ and $0 < j < 2^s$ such that in directory $i$, in sub-table $x^i_0$, in slot $(z_i + j) \bmod 2^{l_i}$, the value is $j + 2^s y_i$.
The total size of slots to check is $d 2^s$.

The total space usage is at most $d 2^w$ pointers, plus $d 2^w$ bits (one for each sub-table to indicate if it is ``small''), plus $\alpha n (s + v + k - w - l)$ bits, where $l$ is the size of the small sub-tables.
This is at most $\alpha n (s + v + k - \lg n + \lg d + 1)$.

%% if that subtable is small, then slot $\lfloor x / 32 \rfloor \bmod 2^{\lfloor \lg n \rfloor - 6}$ has value $\lfloor x / 2^{\lfloor \lg n \rfloor - 1} \rfloor + *y$ or slot $(\lfloor x / 32 \rfloor + 1) \bmod 2^{k-6}$ has value $\lfloor x / 2^{k-1} \rfloor + 2 + 4*y$.
%% if that subtable is large, then slot $\lfloor x / 32 \rfloor$ has value $2*y$ or slot $(\lfloor x / 32 \rfloor + 1) \bmod 2^{k-5}$ has value $1 + 2*y$.




\subsection{Applications}

Dynamic Bloomier filters, \cite{DBLP:journals/corr/abs-cs-0502032}. Any paper citing Raman \& Rao~\cite{raman2003succinct}, or backyard cuckoo hashing~\cite{DBLP:journals/corr/abs-0912-5424}, or De Dictionariis Dynamicis Pauco Spatio Utentibus~\cite{DBLP:journals/corr/abs-cs-0512081}

\section{The tail filter}

The {\em tail filter} is based on the design of the filter in Pagh et al.~\cite{DBLP:journals/corr/abs-1304-1188}

In this section, we'll first describe the abstract semantics of tail filters, using this to show their fpp.
This semantics will equate a tail filter with a set of bitstrings of varying lengths.
After that, we'll describe a mechanism for storing the bitstrings associated with a tail filter in a way that has low space usage and operation time complexity.

\subsection{The abstract semantics of tail filters}

An abstract tail filter $F$ consists of a positive integer $m$, a real number $0 < \epsilon < 1$, and a set of bitstrings of varying lengths. %, and its fpp increases as it grows.
The subset of $U$ it represents, $[F]$, is defined as $\{x \in U : \exists y \in F, y \textrm{ is a prefix of } x\}$.
If any bitstring $x$ in $F$ is a prefix of another bitstring $y$ in $F$, $y$ can be removed from $F$ without changing $[F]$, so wlog, we can assume $F$ is {\em ''prefix-free''}: no bitstring in $F$ is a prefix of any other bitstring in $F$.

% TODO: what about removing bitstrings when a shorter one is inserted? Good news: we only ever insert longer ones

Abstract tail filters have restrictions on the number of strings of each length.
Let $n$ be the number of bitstrings in $F$.
The shortest bitstring has length $\lfloor \lg (m / \epsilon) \rfloor$ and the longest has length $\lfloor \lg (n/m) \rfloor + \lfloor \lg (m / \epsilon) \rfloor$.
This last value is guaranteed to be less than $\lceil 2 \lg |U| \rceil$, since $\epsilon > 1/|U|$.

\begin {itemize}
\item $m$ bitstrings have length $\lfloor \lg (m / \epsilon) \rfloor$
\item For each $0 < j < \lfloor \lg (n/m) \rfloor $, $m2^{j-1}$ bitstrings have length $j + \lfloor \lg (m / \epsilon) \rfloor$.
\item $n - m2^{\lfloor \lg (n/m) \rfloor}$ bitstrings have length $1 + \lfloor \lg (n/m) \rfloor  + \lfloor \lg (m / \epsilon) \rfloor$.
\end {itemize}

The items with a length $j + \lfloor \lg (m / \epsilon) \rfloor$ are said to be in {\em ``cohort''} $j$.
Each bitstring in cohort $j$ adds $2^{- j - \lfloor \lg (m / \epsilon) \rfloor}$ to the fpp, so the full cohort adds $m2^{- \lfloor \lg (m / \epsilon) \rfloor}$ to the fpp.
Each additional cohort $j$, excluding $j = \lfloor \lg (n/m) \rfloor$, adds $m 2^{j-1} 2^{-j - \lfloor \lg (m / \epsilon) \rfloor} = (m/2) 2^{-\lfloor \lg (m / \epsilon) \rfloor}$ to the fpp.
The last cohort adds less than or equal to $(m/2) 2^{-\lfloor \lg (m / \epsilon) \rfloor}$, since $n - m 2^{\lfloor \lg (n/m) \rfloor} \leq m 2^{\lfloor \lg(n/m)\rfloor}$.
The sum total fpp is $m 2^{- \lfloor \lg (m / \epsilon) \rfloor} (1 + \lfloor \lg (n/m) \rfloor)$, which is less than

\begin{equation}
  2 \epsilon (1 + \lfloor \lg (n/m) \rfloor)
\end{equation}

\subsection{Implementing tail filters}

The tail filter implementation tracks, in addition to $m$, $\epsilon$, and the cohort number, two data structures.

One structure stores all bitstrings in $F$ that have length greater than or equal to $\lfloor \lg (n/m) \rfloor + \lfloor \lg m \rfloor$.
They are stored in a map in which the keys are values from $[2^{\lfloor \lg (n/m) \rfloor + \lfloor \lg m \rfloor}]$ and the values are bitstrings of length up to $\lfloor \lg (m/\epsilon) \rfloor - \lfloor \lg m \rfloor$, embedded in bitstrings of length exactly $\lfloor \lg (m/\epsilon) \rfloor - \lfloor \lg m \rfloor + 1$ as described earlier.
Using a quotient DySECT dictionary, this can be stored in $(1 + o(1))n(\lfloor \lg (m/\epsilon) \rfloor - \lfloor \lg m \rfloor + 1) \leq (1+o(1))n(\lg (1/\epsilon) + 2)$ bits of space.

The second structure is present if there are any bitstrings shorter than $\lfloor \lg (n/m) \rfloor + \lfloor \lg m \rfloor$.
Those are embedded in a bitset of length $2^{\lfloor \lg (n/m) \rfloor + \lfloor \lg m \rfloor - 1}$ as follows:
each slot in the bitset is interpreted as a bitstring $s$ of length exactly $\lfloor \lg (n/m) \rfloor + \lfloor \lg m \rfloor - 1$, and is set to {\bf true} if and only if there is any bitstring in $F$ that is a prefix of it.
Note that if there is any bitstring $x \in F$ that is a prefix of a bitstring of length exactly $\lfloor \lg (n/m) \rfloor + \lfloor \lg m \rfloor - 1$, then $x$ has length at most $\lfloor \lg (n/m) \rfloor + \lfloor \lg m \rfloor - 1$.

To perform {\bf lookup$(x)$}, both parts of the tail filter are consulted.
In the dictionary part, the first $\lfloor \lg (n/m) \rfloor + \lfloor \lg m \rfloor$ bits of $x$ are looked up in the map.
If one or more items in the dictionary has that key, each value is inspected to see if it is a prefix of the next $\lfloor \lg(m/\epsilon) \rfloor - \lfloor \lg m \rfloor$ bits of $x$.
If any do, then {\bf lookup} returns {\bf true}.
Otherwise, the bit in the bitset at the location that corresponds to the first $\lfloor \lg (n/m) \rfloor + \lfloor \lg m \rfloor - 1$ bits of $x$ is returned.

To perform {\bf insert$(x)$} on a structure in which cohort $\lfloor \lg (n/m) \rfloor$ is not full, $x$ is inserted into the dictionary: the key is the first $\lfloor \lg (n/m) \rfloor + \lfloor \lg m \rfloor$ bits of $x$ and the value is the next $1 + \lfloor \lg (m/\epsilon) \rfloor - \lfloor \lg m \rfloor$ bits.
If the cohort is full, the tail filter has to be rewritten.
First, the bitset is doubled in size.
Slot $2i + j$ in the new bitset, where $j < 2$, is set to equal slot $i$ in the old bitset.
Second, values from the dictionary with empty tails must be inserted into the bitset.
Finally, the dictionary is rewritten by adding an extra bit to the key portion of each key-value pair, pulled from the value portion.

\subsection{Equivalence of the implementation and the abstract semantics}

We need three facts:

After a call to {\bf insert$(x)$}, {\bf lookup$(x)$} returns {\bf true}.


Rewriting a filter does not change $[F]$. (This implies that neither the fpp nor the results of calls to {\bf lookup} are changed by rewriting a filter.)

Adding a value to the dictionary increases the fpp by $2^{-\lfloor \lg (n/m) \rfloor - \lfloor \lg (m/\epsilon) \rfloor}$.

$F$ and $S$ are said to be ``Equivalent'' when for all $y$, {\bf lookup$(y)$} is true if and only if bitstring in $[F]$ is a prefix of $y$.

{\bf insert$(x, F)$} is equivalent under {\bf lookup} to adding $x$ as a bitstring to $[F]$.

\section{Differences from Pagh et al.}

Doesn't need to double the space when full.

An asymptotically reduced upper bound on the space.

Now oblivious: don't need to know maximum filter size ahead of time.

\bibliography{main}{}
\bibliographystyle{alpha}
\end{document}

%%  LocalWords:  AMQ lookup bitstring bitstrings fpp AMQ's pre DySECT
%%  LocalWords:  quotienting Pagh et al bitset wlog Bloomier Raman
%%  LocalWords:  Rao Dictionariis Dynamicis Pauco Spatio Utentibus
%%  LocalWords:  Fano
